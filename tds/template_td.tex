\documentclass[10pt]{article}
\usepackage[french]{babel}
\usepackage{enumerate}
\usepackage[T1]{fontenc}
\usepackage[utf8]{inputenc}
\usepackage{amsmath,amssymb,amsthm}
\usepackage[top=2cm, bottom=2cm, left=3cm, right=3cm]{geometry}
\usepackage{algorithm}
\usepackage{algpseudocode}
\usepackage{listings}
\usepackage{xcolor}
\usepackage{url}
\author{Gönenç Onay}
\lstset{
  language=Python,
  basicstyle=\ttfamily\small,
  keywordstyle=\color{blue},
  breaklines=true,
  columns=fullflexible,
  frame=single,
  showstringspaces=false,
  keepspaces=true,
  numbers=left,
  numberstyle=\small\color{gray}
}
\usepackage{hyperref}
\newcounter{exocount}
\newenvironment{exo}[1][]
{\refstepcounter{exocount}\par\bigskip\noindent\textbf{Exercice \theexocount.} \ifx&#1&\else (#1)\fi\par\noindent}
{\par\bigskip}

\newtheorem{definition}{Définition}
\newtheorem{innerthm}{Théorème}
\newenvironment{thm}[1][]
{\innerthm\ifx&#1&\else (#1)\fi}
{\endinnerthm}

\newenvironment{preuve}
{\par\noindent\textbf{Preuve.}}
{\hfill$\square$\par\bigskip}

\newcommand{\imrec}[2]{\underleftarrow{#1}{(#2)}}
\date{}
\newcommand{\tq}{\; : \;}
\newcommand{\bb}[1]{\mathbb{#1}}
\newcommand{\entier}[1]{\left\lfloor #1 \right\rfloor}
\title{TD 3: Sous-groupes engendr\'es, morphisms, quotients}
\renewcommand{\epsilon}{\varepsilon}

\begin{document}
\maketitle
\sloppy

\begin{flushright}
	2024 \\
	GSU - Cours MAT-204
\end{flushright}
\hrule
\bigskip
\begin{exo}
	Soit $K \subseteq H \subseteq G$ sous-groupes. Montrer que $[G:K] = [G:H][H:K]$ (la notation $[A : B]$ veut dire \emph{l'indice} de $A$ dans $B$.).
\end{exo}
\begin{exo}
	Soit $p$ un nombre premier et $G$ un groupe d'ordre $p$. Montrer que
	\begin{enumerate}
		\item Les seuls sous-groupes de $G$ sont $\{e_G\}$ et $G$.

		\textbf{Correction :} Soit $H$ un sous-groupe de $G$. Par le théorème de Lagrange, l'ordre de $H$ divise $p$, donc $|H|=1$ ou $|H|=p$. Donc $H = \{e_G\}$ ou $H = G$.

		\item $






	\end{enume
\end{exo}rate}		\textbf{Correction :} Le noyau de $f$ est un sous-groupe de $G$, donc d'ordre $1$ ou $p$. S'il est d'ordre $1$, $f$ est injectif. S'il est d'ordre $p$, alors $\ker f = G$, donc $f$ est trivial.		\item Pour tout groupe $W$, tout homomorphisme $f: G \to W$ est injectif ou trivial (\emph{trivial} veut dire: $f(g)=e_{W}$ pour tout $g \in G$)		\textbf{Correction :} Soit $g \in G$, $g \neq e_G$. Le sous-groupe engendré par $g$ a un ordre qui divise $p$ et est supérieur à $1$, donc $| \langle g \rangle | = p$, donc $\langle g \rangle = G$. Donc $G$ est cyclique.G$ est cyclique.






\begin{exo}
	\begin{enumerate}
		\item Montrer que tout sous-groupe d'un groupe commutative est normal dans $G$.
		\item Trouver un groupe $G$ et $H\subseteq G$, avec $H$ commutative mais qui n'est pas normal dans $G$.
	\end{enumerate}
\end{exo}
\begin{exo}
	Pour $G$ un groupe et $x \in G$, montrer par récurrence que
	$x^{k+\ell} = x^k x^\ell$, pour tout $k$ et $\ell$ dans $\mathbb{N}$. En déduire  que $f:(\mathbb{Z},+) \to \langle x \rangle, k \mapsto x^{k}$ est un homomorphisme surjective. En conclure que $f(k)=e_{G}$ si et seulement si $k$ divise $n$ pour un certain $n \in \mathbb{Z}$ et que $\langle x \rangle$ est isomorphe \`a $\frac{\mathbb{Z}}{n\mathbb{Z}}$.
\end{exo}

\begin{exo}
	Le centre d'un groupe $G$ est l'ensemble $Z(G) := \{x \in G : xg = gx \text{ pour tout } g\in G\}$
	\begin{enumerate}
		\item Montrer que $Z$ est un sous-groupe normal de $G$.
		\item Trouver les centres de $S_3$ et $D_4$.
	\end{enumerate}
\end{exo}
\begin{exo}
	Si $H \leq G$ est un sous-groupe d'indice $2$, montrer qu'il est alors normal dans $G$.
\end{exo}

\begin{exo}
	Trouver des sous-groupes $H$ et $K$ dans $G := S_4$, tels que $H$ soit normal dans $K$ et $K$ soit normal dans $G$, mais $H$ ne soit pas normal dans $G$. Démontrer votre assertion.
	\textbf{Indice :} Considérez le sous-groupe $V_4 := \{Id, (12)(34), (13)(24), (14)(23)\}$ comme l'un des sous-groupes potentiels.
\end{exo}


\end{document}
