\documentclass{beamer}
\usepackage[utf8]{inputenc}
\usepackage[T1]{fontenc}
\usepackage[french]{babel}
\usepackage{csquotes}
\usepackage{amsmath, amssymb, amsfonts}
\usepackage{graphicx}
\usepackage{xcolor}
\usepackage{pgfpages} 
\usepackage{blkarray}
% Personnalisation du thème Beamer
\usetheme{Copenhagen}
\usecolortheme{beaver}
\setbeamerfont{frametitle}{series=\bfseries}
\setbeamerfont{itemize/enumerate subitem}{size=\small}
\setbeamerfont{block title}{size=\normalsize}
\setbeamertemplate{navigation symbols}{}

% Informations de la présentation
\title{\textbf{Les Espaces Métriques : Du Concept à l'Application}}
\subtitle{Un Voyage Abstrait au C\oe ur des Mathématiques Modernes}
\author{Gönenç Onay}
\institute{Université Galatasaray, Département de Mathématiques}
\date{\today}

\begin{document}
\sloppy

% Frame 1: Titre
\frame{\titlepage}

% Frame 2: Introduction - Motivations
\section*{Introduction et Motivations}
\begin{frame}
\frametitle{Introduction : Du Familier à l'Abstrait}
\begin{itemize}
	\item En analyse réelle, la notion de « distance » est intuitive : $d(x,y) = |x-y|$.
	\item La convergence des suites, la continuité des fonctions, et la topologie de $\mathbb{R}$ reposent sur cette idée.
\end{itemize}
\end{frame}

\begin{frame}
	\frametitle{Espaces Complets : La « Fermeture » d'un Espace Métrique}
	\begin{block}{Définition : Espace Complet}
		Un espace métrique $(E, d)$ est \textbf{complet} si toute suite de Cauchy dans $E$ converge vers un point \textit{de} $E$.
	\end{block}
	\begin{itemize}
		\item $\mathbb{R}$ et $\mathbb{R}^n$ (avec la métrique euclidienne) sont complets.
		\item La complétude garantit que l'espace n'a pas de « trous ».
		\item Le processus de complétion permet de « boucher les trous » d'un espace (ex: $\mathbb{Q}$ complété donne $\mathbb{R}$).
	\end{itemize}
\end{frame}

% Frame 11: Complétude et Topologie
\begin{frame}
	\frametitle{Complétude : Démarcation avec les Espaces Purement Topologiques}
	\begin{itemize}
		\item La complétude est une propriété \textbf{métrique}, pas purement topologique.
		      \begin{itemize}
			      \item Un espace homéomorphe à un espace complet n'est pas nécessairement complet.
			      \item Ex : $(0,1)$ est homéomorphe à $\mathbb{R}$ mais n'est pas complet (une suite de Cauchy dans $(0,1)$ peut converger vers $0$ ou $1$).
		      \end{itemize}
		\item C'est une distinction fondamentale :
		      \begin{itemize}
			      \item Les propriétés métriques sont plus « fines » que les propriétés topologiques générales.
			      \item La complétude est indispensable pour des théorèmes clés en analyse (point fixe de Banach, Baire).
		      \end{itemize}
	\end{itemize}
\end{frame}

% Frame 12: Introduction aux applications
\section*{Applications des Espaces Métriques : Un Panorama}
\begin{frame}
	\frametitle{Les Espaces Métriques : Un Langage Unificateur pour l'Innovation}
	\begin{itemize}
		\item Au-delà de l'abstraction, les espaces métriques sont des outils indispensables.
		\item Ils permettent de modéliser, analyser et résoudre des problèmes dans des domaines très variés :
		      \begin{itemize}
			      \item Analyse Fonctionnelle et Théorie de l'Approximation
			      \item Informatique, Théorie des Codes et Machine Learning
			      \item Traitement du signal et des images
			      \item Théorie des Nombres et Physique
		      \end{itemize}
	\end{itemize}
\end{frame}

% Frame 13: Espaces de fonctions et de suites
\subsection*{Analyse Fonctionnelle et Approximation}
\begin{frame}
	\frametitle{Applications : Espaces de Fonctions et de Suites}
	\begin{itemize}
		\item \textbf{Espaces de fonctions continues $C([a,b], \mathbb{R})$} :
		      \begin{itemize}
			      \item Métrique de la convergence uniforme : $d_\infty(f,g) = \sup_{t \in [a,b]} |f(t) - g(t)|$. C'est un espace de Banach (complet).
		      \end{itemize}
		\item \textbf{Espaces $L^p$ (fonctions p-intégrables)} :
		      \begin{itemize}
			      \item Métrique $d_p(f,g) = \left( \int |f(x)-g(x)|^p dx \right)^{1/p}$. Fondamentaux en théorie de l'intégration et EDP.
		      \end{itemize}
		\item \textbf{Espaces de suites $l^p$} : $d_p(x,y) = \left( \sum_{n=0}^\infty |x_n - y_n|^p \right)^{1/p}$.
	\end{itemize}
\end{frame}

% Frame 14a: Stone-Weierstrass (approximation de fonctions)
\begin{frame}
	\frametitle{Théorème de Stone-Weierstrass : Approximation de Fonctions}
	\begin{itemize}
		\item Un espace métrique comme $C([a,b])$ est central pour l'approximation de fonctions.
		\item \textbf{Théorème de Stone-Weierstrass (version réelle)} :
		      \begin{block}{}
			      Soit $K$ un compact et $C(K, \mathbb{R})$ l'espace des fonctions continues de $K$ dans $\mathbb{R}$ muni de la norme $\sup$.\\
			      Tout sous-anneau unitaire séparant les points de $K$ est dense dans $C(K, \mathbb{R})$.
		      \end{block}
		\item \textbf{Intuition} : Les polynômes peuvent approcher n'importe quelle fonction continue sur un compact.
	\end{itemize}
\end{frame}

% Frame 14b: Universal Approximation Theorem (réseaux de neurones)
\begin{frame}
	\frametitle{Théorème d'Approximation Universelle et Réseaux de Neurones}
	\begin{itemize}
		\item \textbf{Extension aux Réseaux de Neurones (Universal Approximation Theorem - UAT)} :
		      \begin{itemize}
			      \item Les réseaux de neurones (sous certaines conditions) peuvent approximer n'importe quelle fonction continue.
			      \item Ceci s'appuie sur des résultats similaires à Stone-Weierstrass, montrant la puissance des structures métriques pour l'apprentissage.
		      \end{itemize}
	\end{itemize}
\end{frame}

% Frame 15: Distance de Hamming
\subsection*{Informatique, Codes et Machine Learning}
\begin{frame}
	\frametitle{Applications : Distance de Hamming et Théorie des Codes}
	\begin{block}{Définition : Distance de Hamming}
		La \textbf{distance de Hamming} entre deux chaînes de caractères (ou vecteurs binaires) de même longueur est le nombre de positions où leurs symboles correspondants sont différents.
	\end{block}
	\begin{itemize}
		\item Exemple : $d_H(\text{« 10110 »}, \text{« 11100 »}) = 2$.
		\item C'est une distance au sens formel.
		\item \textbf{Applications} :
		      \begin{itemize}
			      \item \textbf{Théorie des codes} : Détection et correction d'erreurs dans les transmissions numériques.
			      \item \textbf{Bio-informatique} : Mesure de la divergence génétique entre séquences ADN.
			      \item \textbf{Sécurité} : Comparaison de hachages, vulnérabilités.
		      \end{itemize}
	\end{itemize}
\end{frame}

% Frame 16: Distances en ML
\begin{frame}
	\frametitle{Applications : Distances en Machine Learning}
	\begin{itemize}
		\item \textbf{Clustering et Classification} :
		      \begin{itemize}
			      \item Les algorithmes comme $k$-Means ou $k$-NN reposent sur des distances (Euclidienne, Manhattan) pour regrouper ou classifier des points de données.
		      \end{itemize}
		\item \textbf{Espaces de Caractéristiques (Feature Spaces)} :
		      \begin{itemize}
			      \item Les données (images, textes, sons) sont souvent transformées en vecteurs de caractéristiques. La distance entre ces vecteurs reflète la similarité des objets originaux.
			      \item L'efficacité des modèles ML dépend souvent du choix pertinent de ces distances.
		      \end{itemize}
		\item \textbf{Réduction de dimension} : Des techniques comme t-SNE utilisent des distances pour visualiser des données complexes en basse dimension.
	\end{itemize}
\end{frame}

% Frame 17: Distance de Wasserstein (intuition)
\begin{frame}
	\frametitle{Intuition de la Distance de Wasserstein (Distance du Terre-à-Terre)}
	\begin{itemize}
		\item La distance de Wasserstein, ou « Earth Mover's Distance » (EMD), est une métrique entre distributions de probabilités.
		\item \textbf{Intuition} : C'est le \textbf{coût minimal} pour transformer une distribution de « terre » en une autre, en déplaçant la terre.
		      \begin{itemize}
			      \item Coût = (masse de terre déplacée) $\times$ (distance parcourue).
		      \end{itemize}
	\end{itemize}
\end{frame}

% Frame 18: Wasserstein en ML
\begin{frame}
	\frametitle{Applications de la Distance de Wasserstein en Machine Learning}
	\begin{itemize}
		\item \textbf{Modèles de diffusion (Diffusion Models)} :
		      \begin{itemize}
			      \item Utilisée dans l'optimisation pour des tâches comme la génération d'images à partir de texte.
			      \item Aide à « faire correspondre » le bruit à des images réelles.
		      \end{itemize}
		\item \textbf{Traitement du langage naturel (NLP)} : Pour comparer la sémantique de phrases ou de documents.
		\item \textbf{Imagerie médicale, vision par ordinateur} : Comparaison d'images ou de données multimodales.
	\end{itemize}
\end{frame}

% Frame 19: Corps non-archimédiens
\subsection*{Corps Non-Archimédiens et Analyse p-adique}
\begin{frame}
	\frametitle{Applications : Corps Non-Archimédiens et Analyse p-adique}
	\begin{itemize}
		\item \textbf{Métriques Ultramétriques (non-archimédiennes)} : Satisfait $d(x, z) \le \max(d(x, y), d(y, z))$.
		      \begin{itemize}
			      \item Conséquences surprenantes : tout triangle est isocèle, les boules sont ouvertes et fermées.
		      \end{itemize}
		\item \textbf{Nombres $p$-adiques ($\mathbb{Q}_p$)} : Complétion de $\mathbb{Q}$ par rapport à la valuation $p$-adique.
		      \begin{itemize}
			      \item Offre une perspective différente sur les nombres et la théorie des nombres.
			      \item Chaque entier a une expansion « infinie » dans le sens des $p$-adiques.
		      \end{itemize}
		\item \textbf{Analyse $p$-adique} : Branche de l'analyse sur ces corps.
		      \begin{itemize}
			      \item Applications en théorie des nombres (Théorème de Fermat-Wiles), physique théorique.
		      \end{itemize}
	\end{itemize}
\end{frame}

% Frame 20: Conclusion
\section*{Conclusion}
\begin{frame}
	\frametitle{Conclusion : L'Omniprésence des Espaces Métriques}
	\begin{itemize}
		\item Les espaces métriques sont bien plus qu'une abstraction formelle ; ils sont un \textbf{langage puissant} pour modéliser le monde.
		\item De la convergence des suites à l'intelligence artificielle, en passant par la géométrie et la cryptographie, la notion de distance est fondamentale.
		\item La \textbf{complétude} sépare les espaces purement topologiques des espaces où les outils de l'analyse peuvent être pleinement déployés.
		\item C'est un domaine central des mathématiques, dont la compréhension est essentielle pour aborder des sujets avancés en mathématiques pures et appliquées.
	\end{itemize}
\end{frame}

\end{document}